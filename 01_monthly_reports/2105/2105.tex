\documentclass[twocolumn,a4j]{jsarticle}
\setlength{\topmargin}{-20.4cm}
\setlength{\oddsidemargin}{-10.4mm}
\setlength{\evensidemargin}{-10.4mm}
\setlength{\textwidth}{18cm}
\setlength{\textheight}{26cm}

\usepackage[top=15truemm,bottom=25truemm,left=20truemm,right=20truemm]{geometry}
\usepackage[latin1]{inputenc}
\usepackage{amsmath}
\usepackage{amsfonts}
\usepackage{amssymb}
\usepackage[dvipdfmx]{graphicx}
\usepackage{listings}
\usepackage{listings,jvlisting}
\usepackage{geometry}

\lstset{
basicstyle={\ttfamily},
identifierstyle={\small},
commentstyle={\smallitshape},
keywordstyle={\small\bfseries},
ndkeywordstyle={\small},
stringstyle={\small\ttfamily},
frame={tb},
breaklines=true,
columns=[l]{fullflexible},
xrightmargin=0zw,
xleftmargin=3zw,
numberstyle={\scriptsize},
stepnumber=1,
numbersep=1zw,
lineskip=-0.5ex
}

\makeatletter
\def\@maketitle
{
\begin{center}
{\LARGE \@title \par}
\end{center}
\begin{flushright}
{\large 報告書NO.02\quad\@date\quad\@author}
\end{flushright}
\par\vskip 1.5em
}
\makeatother

\author{来代 勝胤}
\title{令和3年度 5月 報告書}
\date{2021/6/2}

\begin{document}
\columnseprule=0.1mm
\maketitle
\section*{\large 報告内容}
\begin{itemize}
    \item 進捗概要
    \item OpenFOAMでの解析練習
    \item anacondaを利用したプログラミング連取
\end{itemize}
\section{\large 進捗概要}
5月は実験環境における流れ場の数値解析を目標にOpenFOAMを用いて解析練習を行った.
\section{\large OpenFOAMでの解析練習}
\subsection{OpenFOAMにおける解析の流れ}
OpenFOAMを使用した解析は以下のような流れで行った.\\
\subsection{解析の結果}
数値計算を行うことのできたモデル及び条件は以下の3例である.
\renewcommand{\labelenumi}{(\alph{enumi})}
\begin{enumerate}
    \item 回転していないタイヤモデル単体の解析
    \item 回転しているタイヤモデル単体の解析
    \item ホイールハウスを含めた回転していないタイヤモデルの解析
\end{enumerate}
\subsection{今後の課題}
今回は「解析を実行した」のみにとどまってしまったため,
今後は解析結果や解析の条件の分析に努めたい.
分析の内容として,以下のような項目が挙げられる.
\begin{itemize}
    \item 乱流モデルの違いが計算結果に与える影響
    \item 要素数の違いが計算結果及び計算速度に与える影響
    \item メッシュ設定の違いが計算結果に与える影響(境界層・細分化部分の指定 等)
    \item 境界条件の違いによる流れ場への影響
\end{itemize}
\par
分析の方法としては,人間の目では認知できない微妙な違いを判別する必要が
出てくる場合も考えられるため,画像処理を用いて行おうと考えている.
\par
また,数値解析における最大の難点として,計算に長時間必要であることが挙げられるが,
近年では,GPUを用いた並列計算の手法も開発されつつあり,
利用することができれば大幅な計算時間の短縮が期待できる.
長期的な展望としていつか挑戦してみたい.
\subsection{解析における疑問点}
\section{\large opencv 使ってみた}
\end{document}