\documentclass[twocolumn,a4j]{jsarticle}
\setlength{\topmargin}{-20.4cm}
\setlength{\oddsidemargin}{-10.4mm}
\setlength{\evensidemargin}{-10.4mm}
\setlength{\textwidth}{18cm}
\setlength{\textheight}{26cm}

\usepackage[top=15truemm,bottom=25truemm,left=15truemm,right=15truemm]{geometry}
\usepackage[latin1]{inputenc}
\usepackage{amsmath}
\usepackage{amsfonts}
\usepackage{amssymb}
\usepackage[dvipdfmx]{graphicx}
\usepackage[dvipdfmx]{color}
\usepackage{listings}
\usepackage{listings,jvlisting}
\usepackage{geometry}
\usepackage{framed}
\usepackage{color}
\usepackage[dvipdfmx]{hyperref}
\usepackage{ascmac}
\usepackage{enumerate}
\usepackage{tabularx}
\usepackage{cancel}
\usepackage{scalefnt}

\renewcommand{\figurename}{Fig.}
\renewcommand{\tablename}{Table }

\lstset{
basicstyle={\ttfamily},
identifierstyle={\small},
commentstyle={\smallitshape},
keywordstyle={\small\bfseries},
ndkeywordstyle={\small},
stringstyle={\small\ttfamily},
frame={tb},
breaklines=true,
columns=[l]{fullflexible},
xrightmargin=0zw,
xleftmargin=3zw,
numberstyle={\scriptsize},
stepnumber=1,
numbersep=1zw,
lineskip=-0.5ex
}

\makeatletter
\def\@maketitle
{
\begin{center}
{\LARGE \@title \par}
\end{center}
\begin{flushright}
{\large 報告書 NO.06 - 1\quad\@date\quad\@author}
\end{flushright}
\par\vskip 1.5em
}
\makeatother

\setcounter{tocdepth}{3}

\author{来代 勝胤}
\title{令和3年度 10月 第2週 報告書}
\date{2021/10/14}

\begin{document}
\columnseprule=0.1mm

\maketitle
\section*{報告内容}
\begin{enumerate}[1.]
    \item 進捗状況
    \item ロードセルと荷重の関係式の導出
    \item ロードセルとひずみセンサの関係式の導出
    \item ひずみセンサと入力荷重の関係式の導出
    \item 実験データの荷重への換算
\end{enumerate}
\section{進捗状況}
今週は、校正実験データから、以下の手順で\\
ひずみセンサからの出力を実際の入力へと換算を行った。
Cd・Cl値についても算出したが、うまくいかなかった。
\section{ロードセルと荷重の関係式の導出}
2021年6月18日に実施した実験結果より、
ロードセルと荷重の関係式を算出した。\par
ロードセルの出力(横軸)及びロードセルの引張方向に入力した荷重(縦軸)の関係を
表した図は、以下の Fig.1 のようになった。
\begin{figure}[htbp]
    \footnotesize
    \begin{center}
        \includegraphics[width=85mm]{images/02_force&line.png}
        \caption{Loadcell's input and output}
    \end{center}
\end{figure}

また、最小二乗法を用いて近似式を算出し、
$x$をロードセルの出力、$y$を荷重として、
以下の式(1)の結果を得ることができた。
\begin{eqnarray}
    y = 0.12 x - 0.004
\end{eqnarray}

\newpage

\section{ロードセルとひずみセンサの関係式の導出}
2021年6月21日に実施した実験結果より、
ロードセルとひずみセンサの関係式を導出した。\\

\subsection{各センサとロードセルの押込み距離の関係}
実験の操作により、ロードセルをタイヤモデルに接触させ、
x軸(抗力)方向及びy軸(揚力)方向に任意の距離だけ押込み
それぞれロードセル及びタイヤモデルに取り付けられた
2つのひずみセンサのセンサについて測定した。\par
以下の Fig.2 及び Fig.3 はx軸(抗力)方向及びy軸(揚力)方向について、
ロードセルの押込み距離(横軸)と各センサの出力の関係(縦軸)を示している。
\begin{figure}[htbp]
    \footnotesize
    \begin{center}
        \includegraphics[width=85mm]{images/03_length-output_x.png}
        \caption{Correlation between length and output (x-axis)}
        \includegraphics[width=85mm]{images/04_length-output_y.png}
        \caption{Correlation between length and output (y-axis)}
    \end{center}
\end{figure}

\subsection{ロードセルとひずみセンサの出力の関係}
ロードセルの出力(横軸)とタイヤモデルに取り付けられた
ひずみセンサの出力(縦軸)の関係を以下のFig. 4、Fig.5に示す。
\begin{figure}[htbp]
    \footnotesize
    \begin{center}
        \includegraphics[width=85mm]{images/05_strainsensor-loadcell_x.png}
        \caption{Correlation of load-cell and strain-sensors (x-axis)}
        \includegraphics[width=85mm]{images/06_strainsensor-loadcell_y.png}
        \caption{Correlation of load-cell and strain-sensors (y-axis)}
    \end{center}
\end{figure}

\section{ひずみセンサと入力荷重の関係式の導出}
Fig.1,Fig.4 の "drag" 、Fig.5の "lift" の結果から、
ひずみセンサの出力(横軸)と入力荷重(縦軸)の関係について
以下の Fig.6 に示す。\par
\begin{figure}[htbp]
    \footnotesize
    \begin{center}
        \includegraphics[width=85mm]{images/08_strainsensor-forces&line.png}
        \caption{Correlation of input force and strain-sensors}
    \end{center}
\end{figure}
また、drag 及び lift に対して、最小二乗法を用いて近似式を算出し、
$x$をひずみセンサの出力電圧、$y$をタイヤモデルに加わる荷重として、
それぞれ以下の式(2)、式(3)の近似直線を得た。
\begin{eqnarray}
    \mathrm{line 1} \; : \; y &=& -0.143 \; x - 0.10\\
    \mathrm{line 2} \; : \; y &=& 0.303 \; x - 0.07
\end{eqnarray}

\section{実験データの荷重への換算}
作成した近似式を用いて、それぞれのタイヤモデルの実験データから、
出力電圧から推定される荷重について換算を行った。
その結果を以下の Table.2 に示す。\par
なお、Table.1 については換算前のそれぞれのタイヤモデルにおける出力電圧の平均を示している。
\begin{table}[htbp]
    \footnotesize
    \begin{center}
        \caption{Output voltage}
        \includegraphics[width=90mm]{images/Table_1.png}
        \caption{Conversion of voltage to force}
        \includegraphics[width=90mm]{images/Table_2.png}
    \end{center}
\end{table}
\end{document}