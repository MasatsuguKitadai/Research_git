\documentclass[twocolumn,a4j]{jsarticle}
\setlength{\topmargin}{-20.4cm}
\setlength{\oddsidemargin}{-10.4mm}
\setlength{\evensidemargin}{-10.4mm}
\setlength{\textwidth}{18cm}
\setlength{\textheight}{26cm}

\usepackage[top=15truemm,bottom=25truemm,left=20truemm,right=20truemm]{geometry}
\usepackage[latin1]{inputenc}
\usepackage{amsmath}
\usepackage{amsfonts}
\usepackage{amssymb}
\usepackage[dvipdfmx]{graphicx}
\usepackage{listings}
\usepackage{listings,jvlisting}
\usepackage{geometry}
\usepackage{enumerate}

\lstset{
basicstyle={\ttfamily},
identifierstyle={\small},
commentstyle={\smallitshape},
keywordstyle={\small\bfseries},
ndkeywordstyle={\small},
stringstyle={\small\ttfamily},
frame={tb},
breaklines=true,
columns=[l]{fullflexible},
xrightmargin=0zw,
xleftmargin=3zw,
numberstyle={\scriptsize},
stepnumber=1,
numbersep=1zw,
lineskip=-0.5ex
}

\makeatletter
\def\@maketitle
{
\begin{center}
{\LARGE \@title \par}
\end{center}
\begin{flushright}
{\large 報告書NO.03\quad\@date\quad\@author}
\end{flushright}
\par\vskip 1.5em
}
\makeatother

\author{来代 勝胤}
\title{令和3年度 6月 報告書}
\date{2021/7/7}

\begin{document}
\columnseprule=0.1mm
\maketitle
\section*{報告内容}
\begin{itemize}
    \item 進捗概要
    \item 揚抗力測定の事前実験
    \item Arduinoによる温湿度測定
    \item 7月の予定
\end{itemize}
\section{進捗概要}
今月は、揚抗力測定の事前実験を行った。
具体的に行った実験の内容は以下の通りである。
\begin{enumerate}[(1)]
    \item ロードセルの校正実験
    \item タイヤモデルとロードセルの校正実験
\end{enumerate}
また、Arduino講習が実施されていたため、
そこで学んだことを踏まえ、部屋内の温湿度計を製作した。\\
\section{揚抗力測定実験}
\subsection{実験の目的と方法}
作用力測定の実験を行うにあたり、タイヤモデルの上方に設置されているひずみゲージの出力電圧と
実際に加わっている荷重を関係付ける必要がある。
そこで、ロードセルを用いた校正実験を行うことによりそれらを行った。

\begin{enumerate}[Step.1]
    \item ひずみゲージとロードセルの出力電圧を調べる
    \item ひずみゲージの出力電圧と荷重の関係を導く
    \item 導出した関係をもとにひずみゲージの出力電圧から作用力を求める
\end{enumerate}

\subsection{ロードセルの校正実験}


\section{7月の予定}
7月は、引越しの関係もあり、回流水槽を用いた実験ができなくなってしまうため、
タイヤの作用力測定を複数のモデルで行う必要がある。また、大学院入試が近づいてきたため
入試に向けた勉強を優先して行うつもりである。
\end{document}