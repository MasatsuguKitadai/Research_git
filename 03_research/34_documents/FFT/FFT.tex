\documentclass[a4paper]{jsarticle}
\setlength{\topmargin}{-20.4cm}
\setlength{\oddsidemargin}{-10.4mm}
\setlength{\evensidemargin}{-10.4mm}
\setlength{\textwidth}{18cm}
\setlength{\textheight}{26cm}

\usepackage[top=15truemm,bottom=25truemm,left=20truemm,right=20truemm]{geometry}
\usepackage[latin1]{inputenc}
\usepackage{amsmath}
\usepackage{amsfonts}
\usepackage{amssymb}
\usepackage[dvipdfmx]{graphicx}
\usepackage[dvipdfmx]{color}
\usepackage{listings}
\usepackage{listings,jvlisting}
\usepackage{geometry}
\usepackage{framed}
\usepackage{color}
\usepackage[dvipdfmx]{hyperref}
\usepackage{ascmac}
\usepackage{enumerate}
\usepackage{tabularx}
\usepackage{cancel}
\usepackage{scalefnt}

\renewcommand{\figurename}{fig.}
\renewcommand{\tablename}{table }
\newcommand{\redunderline}[1]{\textcolor{BrickRed}{\underline{\textcolor{black}{#1}}}} 

\hypersetup{
	colorlinks=false, % リンクに色をつけない設定
	bookmarks=true, % 以下ブックマークに関する設定
	bookmarksnumbered=true,
	pdfborder={0 0 0},
	bookmarkstype=toc
}

\author{}
\title{高速フーリエ変換}
\date{}

\begin{document}
\maketitle
\section{フーリエ解析}
\begin{itembox}[l]{point}
    \begin{center}
        複雑な関数・現象を\textgt{三角関数}に分解して考える
    \end{center}
\end{itembox}
\subsection{フーリエ級数展開}
関数$f\left(x\right)$の性質を知るために、より\textgt{基本的な関数系}
\begin{eqnarray*}
    \left\{\varphi_0\left(x\right),\varphi_1\left(x\right),\varphi_2\left(x\right),\cdots\right\}
\end{eqnarray*}
で級数展開を考える.
\begin{eqnarray*}
    f\left(x\right)=\sum^\infty_{n=0}C_n\varphi_n\left(x\right)
\end{eqnarray*}
\subsubsection*{$f\left(x\right)$が無限回微分可能 (\textgt{マクローリン展開})}
$\rightarrow\left\{1,x,x^2,x^3,\cdots\right\}$で級数展開
\begin{eqnarray*}
    f\left(x\right) = \sum^\infty_{n=0}\frac{f^{\left(n\right)}\left(0\right)}{n!}\;x^n
\end{eqnarray*}
\subsubsection*{$f\left(x\right)$が周期関数 (\textgt{フーリエ級数展開})}
$\rightarrow\left\{1,\cos x,\sin x,\cos 2x, \sin 2x,\cdots\right\}$ (三角関数系) で級数展開
\begin{eqnarray*}
    f\left(x\right) = a + \sum^\infty_{n=1}\left(a_n \cos nx+b_n \sin nx\right)\;\cdots\; \left(※\right)
\end{eqnarray*}
\subsubsection{三角関数の利点}
$\rightarrow$直交性 (自身以外との内積が0)
\begin{itembox}[l]{関数の内積}
    \begin{eqnarray*}
        f\left(x\right),g\left(x\right)&\;:\;&周期関数 \left(2\pi\right)\\
        \int^\pi_{\pi}f\left(x\right)&&g\left(x\right) dx\\
    \end{eqnarray*}
\end{itembox}
\begin{eqnarray*}
    \int^\pi_{-\pi}\sin mx \;\sin\left(nx\right)dx &=&
    \begin{cases}
        0 \;\left(m\neq n\right) \\
        \pi \left(m=n\right)
    \end{cases}\\
    \int^\pi_{-\pi}\cos mx \;\cos\left(nx\right)dx &=&
    \begin{cases}
        0 \;\left(m\neq n\right) \\
        \pi \left(m=n\right)
    \end{cases}\\
    \int^\pi_{-\pi}\cos mx \;\sin\left(nx\right)dx &=&0\\
\end{eqnarray*}
\subsubsection{定数を求める}
(※)の両辺に$\cos mx \;$をかけて,$\pi \sim -\pi$で積分.\\
\begin{eqnarray*}
    \int^\pi_{-\pi}f\left(x\right)\cos mx \; dx
    &=& \int^\pi_{-\pi}a\cos  mx \; dx + \sum^\infty_{n=1}\int^\pi_{-\pi}\left(a_n \cos\left(nx\right) \cos mx \; + b_n \sin \left(nx\right) \cos mx \;\right) dx\\
    &=&a_m \pi
\end{eqnarray*}
(※)の両辺に$\sin mx \;$をかけて,$\pi \sim -\pi$で積分.\\
\begin{eqnarray*}
    \int^\pi_{-\pi}f\left(x\right)\sin mx \; dx
    &=& \int^\pi_{-\pi}a\sin mx \; dx + \sum^\infty_{n=1}\int^\pi_{-\pi}\left(a_n \cos\left(nx\right) \sin mx + b_n \sin \left(nx\right) \sin mx \right) dx\\
    &=&b_m \pi
\end{eqnarray*}
(※)の両辺を$\pi \sim -\pi$で積分.\\
\begin{eqnarray*}
    \int^\pi_{-\pi}f\left(x\right) dx= 2\pi a
\end{eqnarray*}
上記の式より
\begin{eqnarray*}
    a_m &=& \frac{1}{\pi} \int^\pi_{-\pi}f\left(x\right)\cos mx \; dx\\
    b_m &=& \frac{1}{\pi} \int^\pi_{-\pi}f\left(x\right)\sin mx \; dx\\
    a &=& \frac{1}{2\pi} \int^\pi_{-\pi}f\left(x\right)dx\\
\end{eqnarray*}
$a_m$の式に$m=0$を代入すると
\begin{eqnarray*}
    a_0 = \frac{1}{\pi} \int^\pi_{-\pi}f\left(x\right) dx = 2a\\
\end{eqnarray*}
$\rightarrow$ 以後、$a$を$\dfrac{a_0}{2}$として,(※)は以下のように書き表せる.
\begin{itembox}[l]{$f\left(x\right)$のフーリエ級数}
    \begin{eqnarray*}
        \frac{a_0}{2} + \sum^\infty_{n=1}\left(a_n \cos nx+b_n \sin nx\right)\\ \\
        \begin{cases}
            a_n = \displaystyle \frac{1}{\pi} \int^\pi_{-\pi}f\left(x\right)\cos nx \; dx \\
            b_n = \displaystyle \frac{1}{\pi} \int^\pi_{-\pi}f\left(x\right)\sin nx \; dx \\
        \end{cases}\\
    \end{eqnarray*}
    \begin{center}
        これを、\textgt{$f\left(x\right)$のフーリエ級数}という.\\
        $f\left(x\right)$に収束するかどうか($f\left(x\right)$と一致するか)は別問題.
    \end{center}
\end{itembox}
\subsubsection{フーリエ級数の収束性}
周期$2\pi$の周期関数$f\left(x\right)$が\textgt{区分的に滑らか}であるとき、$f\left(x\right)$のフーリエ級数は以下のように収束する。
\begin{itembox}[l]{フーリエ級数の収束}
    \begin{eqnarray*}
        \frac{a_0}{2}+\sum^{\infty}_{n=1}\left(a_n\cos\left(nx\right)+b_n\sin\left(nx\right)\right) =
        \begin{cases}
            f\left(x\right) \; (連続な点)                                              \\
            \\
            \displaystyle\frac{f\left(x-0\right)+f\left(x+0\right)}{2} \; (不連続な点) \\
        \end{cases}\\
    \end{eqnarray*}
\end{itembox}
\subsubsection*{区分的に滑らか}
$a \leq x \leq b$上の関数$f\left(x\right)$が次の条件を満たすとき,$a \leq x \leq b$で\textgt{区分的に連続}であるという。
\begin{enumerate}[(i)]
    \item $a \leq x \leq b$で有限個の点を除いて連続
    \item 不連続点$c$において$f\left(c-0\right),f\left(c+0\right)$が存在する。
\end{enumerate}
さらに、導関数$f'\left(x\right)$が\textgt{区分的に連続}であるとき、関数$f\left(x\right)$を\textgt{区分的に滑らか}という。

\subsection{複素フーリエ級数}
\begin{itembox}[l]{オイラーの公式}
    \begin{eqnarray*}
        e^{i\theta} = \cos \theta + i \sin \theta\\
    \end{eqnarray*}
\end{itembox}
\begin{eqnarray*}
    \begin{cases}
        e^{in\theta} = \cos n\theta + i \sin n\theta  \\
        e^{-in\theta} = \cos n\theta - i \sin n\theta \\
    \end{cases}
    \; より \;
    \begin{cases}
        \displaystyle\cos nx = \frac{ e^{in\theta} +  e^{-in\theta}}{2}  \\
        \displaystyle\cos nx = \frac{ e^{in\theta} -  e^{-in\theta}}{2i} \\
    \end{cases}
\end{eqnarray*}
これをフーリエ級数に代入して整理すると、以下のように表すことができる.
\begin{eqnarray*}
    \frac{a_0}{2} + \sum^\infty_{n=1}\left(a_n \cos nx+b_n \sin nx\right)
    &=&\frac{a_0}{2} + \sum^\infty_{n=1}\left(\frac{a_n-ib_n}{2}e^{inx}+\frac{a_n+ib_n}{2}e^{-inx}\right)\\
    &=&\sum^\infty_{n=-\infty} C_n e^{inx}\\
    \begin{cases}
        \displaystyle C_0 = \frac{a_0}{2}         \\
        \displaystyle C_n = \frac{a_n-ib_n}{2}    \\
        \displaystyle C_{-n} = \frac{a_n+ib_n}{2} \\
    \end{cases}
    &&とする
\end{eqnarray*}
\begin{itembox}[l]{複素フーリエ級数}
    \begin{eqnarray*}
        複素フーリエ級数 &:& \sum^\infty_{n=-\infty} C_n e^{inx} \\
        複素フーリエ係数 &:& C_n = \frac{1}{2\pi} \int^{\pi}_{-\pi} f\left(x\right) e^{-inx} \;dx\\
    \end{eqnarray*}
\end{itembox}
\subsubsection*{$C_n$の性質}
$f\left(x\right)$が実関数であるとき $C_{-n}=C_n*$が成立する。($C_n^*$は$C_n$の複素共役)\\
$\rightarrow$ $n$が非負なものを計算すれば求められる。
\begin{eqnarray*}
    C_{-n} &=& \frac{1}{2\pi} \int^{\pi}_{-\pi} f\left(x\right) e^{inx} \;dx\\
    C_{n}^* &=& \frac{1}{2\pi} \int^{\pi}_{-\pi} f\left(x\right) \left(e^{inx}\right)^* \;dx\\
    &=& \frac{1}{2\pi} \int^{\pi}_{-\pi} f\left(x\right) e^{inx} \;dx\\
    &=& C_{-n}
\end{eqnarray*}

\subsection{一般周期のフーリエ級数}
\begin{itembox}[l]{point}
    \begin{center}
        周期$2\pi$の関数へと変換する
    \end{center}
\end{itembox}
周期が$2L$の関数を考える.
\begin{eqnarray*}
    f\left(x\right) \;&:&\; 周期 \; 2L\\
    &\downarrow& \; x = \frac{L}{\pi} t \; として変数変換\\
    f\left(\frac{L}{\pi} t\right) \;&:&\; 周期 \; 2\pi\\
    f\left(\frac{L}{\pi} t\right) &=& h\left(t\right) \; とする
\end{eqnarray*}
\begin{center}
    グラフは横軸方向に $\dfrac{\pi}{L}$ 倍される
\end{center}

$h\left(t\right)$をフーリエ級数展開すると、
\begin{eqnarray*}
    &&h\left(t\right) \sim \frac{a_0}{2}+\sum^{\infty}_{n=1}\left(a_n \cos nt + b_n \sin nt\right)\\
    &&\begin{cases}
        a_n=\displaystyle\frac{1}{\pi}\int^{\pi}_{-\pi}h\left(t\right)\cos nt dt \\
        b_n=\displaystyle\frac{1}{\pi}\int^{\pi}_{-\pi}h\left(t\right)\sin nt dt
    \end{cases}
\end{eqnarray*}
$t$を$x$に戻すと
\begin{eqnarray*}
    &&f\left(x\right) \sim \frac{a_0}{2}+\sum^{\infty}_{n=1}\left(a_n \cos \frac{n\pi}{L}x + b_n \sin \frac{n\pi}{L}x\right)\\
    &&\begin{cases}
        a_n=\displaystyle\frac{1}{L}\int^{L}_{-L}f\left(x\right)\cos \frac{n\pi}{L}x dx \\
        b_n=\displaystyle\frac{1}{L}\int^{L}_{-L}f\left(x\right)\sin \frac{n\pi}{L}x dx
    \end{cases}
\end{eqnarray*}

複素数の場合
\begin{eqnarray*}
    f\left(x\right) &\sim& \sum^{\infty}_{n=-\infty} c_n \exp{\left(i \frac{n\pi}{L}x\right)}\\
    c_n &=& \displaystyle\frac{1}{2L}\int^{L}_{-L}f\left(x\right) \exp{\left(-i \frac{n\pi}{L}x\right)} dx
\end{eqnarray*}

\subsection{フーリエ変換}

\subsubsection{フーリエの積分公式}
\begin{itembox}[l]{point}
    \begin{center}
        非周期関数 $\rightarrow$ 周期 $\infty$ として考える.
    \end{center}
\end{itembox}

はじめに、周期$2L$の関数$f\left(x\right)$のフーリエ級数展開を考える。
\begin{eqnarray*}
    \begin{cases}
        \displaystyle f\left(x\right) \sim \sum^{\infty}_{n=-\infty} c_n \exp{\left(i \frac{n\pi}{L}x\right)} \\
        \displaystyle c_n = \displaystyle\frac{1}{2L}\int^{L}_{-L}f\left(t\right) \exp{\left(-i \frac{n\pi}{L}x\right)} dx
    \end{cases}
\end{eqnarray*}
ここで、$c_n$をそのまま$f\left(x\right)$に代入すると、
\begin{eqnarray*}
    f\left(x\right) &\sim& \sum^{\infty}_{n=-\infty} \left[ \frac{1}{2L}\int^{L}_{-L}f\left(t\right) \exp{\left(-i \frac{n\pi}{L}t\right)} dt\right] \exp{\left(i \frac{n\pi}{L}x\right)}\\
    \omega_n &=& \frac{n\pi}{L}\\
    \Delta \omega &=& \omega_{n+1} -\omega_n = \frac{\pi}{L}
\end{eqnarray*}
書き換えると,
\begin{eqnarray*}
    f\left(x\right) &\sim& \sum^{\infty}_{n=-\infty} \left[ \frac{1}{2\pi}\int^{L}_{-L}f\left(t\right) \exp{\left(-i \omega_n  t\right)} dt\right] \exp{\left(i \omega_n x \right)} \Delta \omega
\end{eqnarray*}
また、$G\left(\omega_n\right)$ を,以下のようにおく。
\begin{eqnarray*}
    G\left(\omega_n\right) &=&  \frac{1}{2\pi}\int^{L}_{-L}f\left(t\right) \exp{\left(-i \omega_n t \right)} dt\\
\end{eqnarray*}
ここで、$L \rightarrow \infty$で$\Delta \omega \rightarrow 0$より、
\begin{eqnarray*}
    \lim_{\Delta \omega \rightarrow 0} f\left(x\right)
    &=& \lim_{\Delta \omega \rightarrow 0} \sum_{n=-\infty}^{\infty} G\left(\omega_n\right) \Delta \omega\\
    &=& \int^{\infty}_{-\infty} G\left(\omega\right) d\omega\\
\end{eqnarray*}
\begin{itembox}[l]{フーリエの積分公式}
    \begin{eqnarray*}
        f\left(x\right) \sim \int^{\infty}_{-\infty}\left[ \frac{1}{2\pi} \int^{\infty}_{-\infty}f\left(t\right)\exp\left(-i \omega t\right)\right] \exp\left(i \omega x\right) d\omega
    \end{eqnarray*}
    \begin{center}
        ※成立する条件 : $f\left(x\right)$が$\left(-\infty,\infty\right)$で区分的に滑らかで,$\left(-\infty,\infty\right)$において\textgt{絶対可積分}\\
    \end{center}
\end{itembox}
\begin{itembox}[l]{絶対可積分}
    \begin{eqnarray*}
        \int^{\infty}_{-\infty}\left| f\left(x\right)\right| dx < \infty\\
    \end{eqnarray*}
\end{itembox}

\subsubsection{フーリエ変換}
\begin{eqnarray*}
    f\left(x\right) &\sim& \frac{1}{2\pi}\int^{\infty}_{-\infty}\left[\int^{\infty}_{-\infty}f\left(t\right)\exp \left(-i \omega t\right)dt\right] \exp \left(i\omega x\right) d\omega\\
    \\
    &&ここで\\
    \\
    F\left(\omega\right) &=& \int^{\infty}_{-\infty}f\left(t\right)\exp \left(-i \omega t\right)dt \\
    \\
    &&として,元の式に代入すると\\
    \\
    f\left(x\right) &\sim& \frac{1}{2\pi}\int^{\infty}_{-\infty} F\left(\omega\right) \exp \left(i\omega x\right) d\omega\\
\end{eqnarray*}
\begin{itembox}[l]{フーリエ変換}
    \begin{eqnarray*}
        F\left(\omega\right) &=& \int^\infty_{-\infty} f\left(t\right) \exp\left(-i\omega t\right)\\
        &=& F\left[f\left(t\right)\right]
    \end{eqnarray*}
    \begin{center}
        ※ 絶対加積分であれば$F\left(\omega\right)$ は存在
    \end{center}
\end{itembox}
\subsection{フーリエの反転公式と逆フーリエ変換}
\begin{itembox}[l]{フーリエの反転公式}
    \begin{eqnarray*}
        F\left(\omega\right)&=&F\left[f\left(t\right)\right]\;とすると、\\
        \\
        \frac{1}{2\pi}\int^{\infty}_{-\infty} F\left(\omega\right) \exp \left(i\omega x\right) d\omega &=&
        \begin{cases}
            f\left(x\right) \; \left(連続な点\right)                                  \\
            \\
            \dfrac{f\left(x-0\right)+f\left(x+0\right)}{2} \; \left(不連続な点\right) \\
        \end{cases}\\
    \end{eqnarray*}
\end{itembox}
\begin{itembox}[l]{フーリエ逆変換}
    \begin{eqnarray*}
        F\left(\omega\right)&=&F\left[f\left(t\right)\right]\;とすると、\\
        \\
        F^{-1}\left[F\left(\omega\right)\right] &:=&  \frac{1}{2\pi}\int^{\infty}_{-\infty} F\left(\omega\right) \exp \left(i\omega x\right) d\omega\\
    \end{eqnarray*}
\end{itembox}

\subsection{まとめ}
\begin{itembox}[l]{point}
    \begin{eqnarray*}
        ・フーリエ級数 &\rightarrow& 周期関数を波の合成で表す  \rightarrow ある程度の周波数の波で事足りる\\
        ・フーリエ変換 &\rightarrow& 非周期関数を波の合成で表す \rightarrow 無数の連続的な周波数の波が必要\\
    \end{eqnarray*}
\end{itembox}

\section{離散フーリエ変換 (dicrete Fourier transform)}
\subsection{離散フーリエ変換とは}
信号処理等の離散化されたデジタル信号の周波数解析等によく利用される。
また、偏微分方程式や畳み込み積分の数値計算を効率的に行うことに利用される。
離散フーリエ変換は、計算機上で後述の高速フーリエ変換(FFT)を使って高速に計算することができる。
\begin{itembox}[l]{離散フーリエ変換}
    \begin{eqnarray*}
        F\left(t\right) &=& \sum^{N-1}_{x-} f\left(x\right) \exp \left(-i \frac{2\pi tx}{N}\right)\\
        N\; &:&\; 任意の自然数\\
    \end{eqnarray*}
\end{itembox}
\begin{itembox}[l]{離散フーリエ変換 (2)}
    \begin{eqnarray*}
        F\left(\frac{n}{\Delta t N}\right) &=& \sum^{N-1}_{k=0} \Delta t\cdot f\left(k\cdot \Delta t\right) \exp \left(-i \frac{2\pi nk}{N}\right)\\
        \\
        N\; &:&\; 値の数\\
        \Delta t \;&:&\; サンプリング周期\\
        \frac{1}{\Delta t N} \;&:&\; 基本周波数\\
        k \;&:&\; 時間領域において何番目の数値か\\
        n \;&:&\; 基本周数の何倍か\\
    \end{eqnarray*}
\end{itembox}

\subsection{逆離散フーリエ変換 (inverse discrete Fourier transform)}
\begin{itembox}[l]{逆離散フーリエ変換}
    \begin{eqnarray*}
        f\left(x\right) = \frac{1}{N}\sum ^{N-1}_{t=0} F\left(t\right) \exp \left(i\frac{2\pi xt}{N}\right)\\
    \end{eqnarray*}
\end{itembox}

\section{高速フーリエ変換(fast Fourier transform)}
\subsection{高速フーリエ変換とは}
\textgt{高速フーリエ変換 (fast Fourier transform)}とは、
離散フーリエ変換を計算機上で高速に計算するアルゴリズムである。
\begin{itembox}[l]{高速フーリエ変換}
    \begin{eqnarray*}
        F\left(t\right) = \sum^{N-1}_{x=0}f\left(x\right) \exp\left(-i\frac{2 \pi tx}{N}\right)\\
    \end{eqnarray*}
\end{itembox}

\section{用語}
\subsubsection*{バッファ}
バッファとは、データが 1 つの場所から別の場所に転送する際に、データを一時的に保持するメモリストレージ領域のことをいいます。\\
プログラムがデータをバッファに書き込もうとした際に、データの量がメモリバッファのストレージ容量を超えてしまうと、隣接するメモリ位置を上書きしてしまいます。\\
この結果発生する事象をバッファオーバーフロー(バッファオーバーラン)と呼びます。\\
\section{ファイルの読み書き}

\begin{itembox}[l]{ファイルへの読み書き}
    \begin{enumerate}[(1)]
        \item 1文字の読み込み\\
              int fgetc(FILE *fp);
        \item 1文字の書き出し\\
              int fputc(int c, FILE *fp);
        \item 文字列の書き出し\\
        \item int fprintf(FILE *steram, char *format, ...);
    \end{enumerate}
\end{itembox}
\end{document}