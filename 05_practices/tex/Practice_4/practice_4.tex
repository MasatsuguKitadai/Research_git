\documentclass[12pt,a4paper]{jsarticle}

\usepackage[latin1]{inputenc}
\usepackage{amsmath}
\usepackage{amsfonts}
\usepackage{amssymb}
\usepackage[dvipdfmx]{graphicx}
\usepackage{listings}
\usepackage{listings,jvlisting}
\usepackage{geometry}

\lstset{
basicstyle={\ttfamily},
identifierstyle={\small},
commentstyle={\smallitshape},
keywordstyle={\small\bfseries},
ndkeywordstyle={\small},
stringstyle={\small\ttfamily},
frame={tb},
breaklines=true,
columns=[l]{fullflexible},
xrightmargin=0zw,
xleftmargin=3zw,
numberstyle={\scriptsize},
stepnumber=1,
numbersep=1zw,
lineskip=-0.5ex
}

\author{来代 勝胤}
\title{PIV計測 (1) -2}

\begin{document}
\maketitle
\thispagestyle{empty}
\clearpage
\addtocounter{page}{-1}

\newgeometry{left=15mm,right=15mm,top=15mm,bottom=20mm}

\begin{flushleft}
    {\large \textbf{【演習1】}\\
        格子点$\left(i,j\right)$における渦度$\omega_z$$_i$$_,$$_j$を速度$u$$_i$$_,$$_j$,$v$$_i$$_,$$_j$を用いて,
        中心差分で差分表示する.\\
    }
\end{flushleft}
格子点$\left(i,j\right)$における$x$方向速度成分を$u$$_i$$_,$$_j$としたとき,
            速度$u _{\left( i,j+1 \right)}$,$u _{\left( i,j-1 \right)}$をテーラー展開を用いて表すと,以下のようになる.\\
            \begin{eqnarray}
                \begin{aligned}
                    u_{\left( i , j+1 \right)}
                     & = u_{\left( i, j+\Delta y \right)}                                                 \\
                     & = u_{\left( i , j \right)} + \Delta y \left( \frac{\partial u}{\partial y} \right)
                    + \frac{\Delta y^2}{2!} \left( \frac{\partial^2 u}{\partial y^2}\right)
                    + \frac{\Delta y^3}{3!} \left( \frac{\partial^3 u}{\partial y^3}\right) + ...         \\
                     & = u_{\left( i , j \right)}
                    + \Delta y \left( \frac{\partial u}{\partial y} \right)
                    + \frac{\Delta y^2}{2!} \left( \frac{\partial^2 u}{\partial y^2}\right)
                    + \rm{O}\left( \Delta \it{y^3} \right)
                \end{aligned}
            \end{eqnarray}
            \\
            \begin{eqnarray}
                \begin{aligned}
                    u_{\left( i , j-1 \right)}
                     & = u_{\left( i, j-\Delta y \right)}                                                 \\
                     & = u_{\left( i , j \right)} - \Delta y \left( \frac{\partial u}{\partial y} \right)
                    + \frac{\Delta y^2}{2!} \left( \frac{\partial^2 u}{\partial y^2}\right)
                    - \frac{\Delta y^3}{3!} \left( \frac{\partial^3 u}{\partial y^3}\right) + ...         \\
                     & = u_{\left( i , j \right)}
                    - \Delta y \left( \frac{\partial u}{\partial y} \right)
                    + \frac{\Delta y^2}{2!} \left( \frac{\partial^2 u}{\partial y^2}\right)
                    - \rm{O} \left( \Delta \it{y^3} \right)                                               \\
                \end{aligned}
            \end{eqnarray}
            式(1),(2)より,
            \begin{eqnarray}
                u_{\left( i , j+1 \right)} - u_{\left( i , j-1\right)}
                & = & 2 \Delta y \left( \frac{ \partial u}{\partial y} \right) + \rm{O} \left( \Delta \it{y} \right) \\
                \left( \frac{ \partial u}{\partial y} \right)
                & = & \frac{1}{2 \Delta y} \left\{ u_{\left( i , j+1 \right))} - u_{\left( i , j-1\right)} \right\}
            \end{eqnarray}\\
            同様に,格子点$\left(i,j\right)$における$y$方向速度成分を$v$$_i$$_,$$_j$としたとき,速度$v _{\left( i+1,j \right)}$,$v _{\left( i-1,j \right)}$
をテーラー展開を用いて表すと,以下のようになる.
\begin{eqnarray}
    v_{\left( i+1 , j \right)} - v_{\left( i-1 , j\right)}
    & = & 2 \Delta x \left( \frac{ \partial v}{\partial x} \right) + \rm{O} \left( \Delta \it{x} \right) \\
    \left( \frac{ \partial v}{\partial x} \right)
    & = & \frac{1}{2 \Delta x} \left\{ v_{\left( i+1 , j \right))} - v_{\left( i-1 , j \right)} \right\}
\end{eqnarray}\\

\begin{flushleft}
    {\large \textbf{【Programs】}
    }
\end{flushleft}
\small
\begin{lstlisting}
\end{lstlisting}

\end{document}