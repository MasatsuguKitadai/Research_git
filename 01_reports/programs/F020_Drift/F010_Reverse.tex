\documentclass[12pt,a4paper]{jsarticle}

\setlength{\topmargin}{-20.4cm}
\setlength{\oddsidemargin}{-10.4mm}
\setlength{\evensidemargin}{-10.4mm}
\setlength{\textwidth}{18cm}
\setlength{\textheight}{26cm}
\renewcommand{\figurename}{fig.}
\renewcommand{\tablename}{table }

\usepackage[top=15truemm,bottom=25truemm,left=20truemm,right=20truemm]{geometry}
\usepackage[latin1]{inputenc}
\usepackage{amsmath}
\usepackage{amsfonts}
\usepackage{amssymb}
\usepackage[dvipdfmx]{graphicx}
\usepackage{listings}
\usepackage{listings,jvlisting}
\usepackage{geometry}

\lstset{
  basicstyle={\ttfamily},
  identifierstyle={\small},
  commentstyle={\smallitshape},
  keywordstyle={\small\bfseries},
  ndkeywordstyle={\small},
  stringstyle={\small\ttfamily},
  frame={tb},
  breaklines=true,
  columns=[l]{fullflexible},
  xrightmargin=0zw,
  xleftmargin=3zw,
  numberstyle={\scriptsize},
  stepnumber=1,
  numbersep=1zw,
  lineskip=-0.5ex
}

\author{}
\title{}

\begin{document}

\scriptsize{

\begin{lstlisting}

    int Reverse(char date[], char angle[])
    {
        /*****************************************************************************/
    
        // ディレクトリ文字列作成
        char directoryname[100];
        char directoryname_csv[100];
        char directoryname_dat[100];
        char directoryname_png[100];
    
        /*****************************************************************************/
    
        // 元ディレクトリの作成
        sprintf(directoryname, "../Result/%s", date);
        mkdir(directoryname, mode);
    
        /*****************************************************************************/
    
        // ディレクトリの作成
        sprintf(directoryname, "../Result/%s/010_Reverse", date);
        mkdir(directoryname, mode);
    
        /*****************************************************************************/
    
        // ディレクトリの作成
        sprintf(directoryname_dat, "../Result/%s/010_Reverse/dat", date);
        sprintf(directoryname_csv, "../Result/%s/010_Reverse/csv", date);
        sprintf(directoryname_png, "../Result/%s/010_Reverse/png", date);
    
        mkdir(directoryname_dat, mode);
        mkdir(directoryname_csv, mode);
        mkdir(directoryname_png, mode);
    
        /*****************************************************************************/
    
        // ファイルの指定
        char filename_read[100];
        char filename_csv[100];
        char filename_dat[100];
    
        sprintf(filename_read, "../Data/%s/%s.csv", date, angle);
        sprintf(filename_csv, "../Result/%s/010_Reverse/csv/%s.csv", date, angle);
        sprintf(filename_dat, "../Result/%s/010_Reverse/dat/%s.dat", date, angle);
    
        /*****************************************************************************/
    
        // 変数宣言
        int i, j;
        int datalength = 0;
        int ch = 3;
        double value[5000][ch];
        double ch0, ch1, ch2; // ch0:drag, ch1:lift, ch2:load-cell
    
        double time = 0;
    
        // 配列の初期化
    
        for (i = 0; i < 5000; i++)
        {
            for (j = 0; j < 3; j++)
            {
                value[i][j] = 0;
            }
        }
    
        // printf("check");
    
        // ファイルの読み込み
        fp = fopen(filename_read, "r");
        if (fp == NULL)
        {
            printf("01\t[%s]\tno data file\n", angle);
            return 1;
        }
    
        i = 0;
    
        while ((fscanf(fp, "%lf, %lf, %lf", &ch0, &ch1, &ch2)) != EOF)
        {
            // printf("%.3f, %.3f, %.3f\n", ch0, ch1, ch2);
            value[i][0] = ch0;
            value[i][1] = ch1;
            value[i][2] = ch2;
            i = i + 1;
        }
    
        fclose(fp);
    
        datalength = i;
    
        // 計算
    
        for (i = 0; i < datalength; i++)
        {
            value[i][2] = -1 * value[i][2];
        }
    
        fp_csv = fopen(filename_csv, "w");
        fp_dat = fopen(filename_dat, "w");
    
        for (i = 0; i < datalength; i++)
        {
            time = i * 0.2;
            fprintf(fp_csv, "%.3f,%.3f,%.3f\n", value[i][0], value[i][1], value[i][2]);
            fprintf(fp_dat, "%d\t%.3f\t%.3f\t%.3f\t%.1f\n", i, value[i][0], value[i][1], value[i][2], time);
        }
    
        fclose(fp_csv);
        fclose(fp_dat);
    
        /*****************************************************************************/
        // Gnuplot //
    
        // ディレクトリの作成
        char directoryname_png_1[100];
        char directoryname_png_2[100];
        char directoryname_png_3[100];
    
        sprintf(directoryname_png_1, "../Result/%s/010_Reverse/png/Loadcell", date);
        sprintf(directoryname_png_2, "../Result/%s/010_Reverse/png/Strainsensors", date);
        sprintf(directoryname_png_3, "../Result/%s/010_Reverse/png/Allsensors", date);
    
        mkdir(directoryname_png_1, mode);
        mkdir(directoryname_png_2, mode);
        mkdir(directoryname_png_3, mode);
    
        /*****************************************************************************/
    
        // filename
        char filename_png_1[100];
        char filename_png_2[100];
        char filename_png_3[100];
    
        sprintf(filename_dat, "../Result/%s/010_Reverse/dat/%s.dat", date, angle);
        sprintf(filename_png_1, "../Result/%s/010_Reverse/png/Loadcell/%s.png", date, angle);
        sprintf(filename_png_2, "../Result/%s/010_Reverse/png/Strainsensors/%s.png", date, angle);
        sprintf(filename_png_3, "../Result/%s/010_Reverse/png/Allsensors/%s.png", date, angle);
    
        /*****************************************************************************/
    
        // range x
        int x_min = 0;
        int x_max = 600;
    
        // range y
        double y_min = -1;
        double y_max = 1;
    
        // range y (loadcell)
        double y_min_loadcell = -0.5;
        double y_max_loadcell = 2.5;
    
        // label
        const char *xxlabel = "Time [s]";
        const char *yylabel = "Output voltage [V]";
        char label[100];
    
        double size;
    
        double angle_2 = 0;
    
        angle_2 = atoi(angle);
        angle_2 = angle_2;
        sprintf(label, "%0.f", angle_2);
    
        // size
        size = 1;
    
        /*****************************************************************************/
    
        if ((gp = popen("gnuplot", "w")) == NULL)
        {
            printf("gnuplot is not here!\n");
            exit(0); // gnuplotが無い場合、異常ある場合は終了
        }
    
        fprintf(gp, "set terminal pngcairo enhanced font 'Times New Roman,15' \n");
        fprintf(gp, "set output '%s'\n", filename_png_1);
        // fprintf(gp, "set multiplot\n");
        fprintf(gp, "unset key\n");
        fprintf(gp, "set term pngcairo size 1280, 960 font ',27'\n");
        // fprintf(gp, "set size ratio %.3f\n", size);
    
        fprintf(gp, "set lmargin screen 0.10\n");
        fprintf(gp, "set rmargin screen 0.90\n");
        fprintf(gp, "set tmargin screen 0.90\n");
        fprintf(gp, "set bmargin screen 0.15\n");
    
        fprintf(gp, "set xrange [%d:%d]\n", x_min, x_max);
        fprintf(gp, "set xlabel '%s'offset 0.0,0\n", xxlabel);
        fprintf(gp, "set yrange [%.3f:%.3f]\n", y_min_loadcell, y_max_loadcell);
        fprintf(gp, "set ylabel '%s'offset 1.0,0.0\n", yylabel);
        fprintf(gp, "set title 'Load cell : %s [deg]'\n", label);
    
        // fprintf(gp, "set samples 10000\n");
        fprintf(gp, "plot '%s' using 5:4 with lines lc 'black' title 'loadcell'\n", filename_dat);
        fflush(gp); // Clean up data
    
        /*****************************************************************************/
    
        fprintf(gp, "set terminal pngcairo enhanced font 'Times New Roman,15' \n");
        fprintf(gp, "set output '%s'\n", filename_png_2);
        // fprintf(gp, "set multiplot\n");
        fprintf(gp, "set key left top\n");
        fprintf(gp, "set key font ',22'\n");
        fprintf(gp, "set term pngcairo size 1280, 960 font ',27'\n");
        // fprintf(gp, "set size ratio %.3f\n", size);
    
        fprintf(gp, "set lmargin screen 0.10\n");
        fprintf(gp, "set rmargin screen 0.90\n");
        fprintf(gp, "set tmargin screen 0.90\n");
        fprintf(gp, "set bmargin screen 0.15\n");
    
        fprintf(gp, "set xrange [%d:%d]\n", x_min, x_max);
        fprintf(gp, "set xlabel '%s'offset 0.0,0\n", xxlabel);
        fprintf(gp, "set yrange [%.3f:%.3f]\n", y_min, y_max);
        fprintf(gp, "set ylabel '%s'offset 1.0,0.0\n", yylabel);
        fprintf(gp, "set title 'Strain sensors : %s [deg]'\n", label);
    
        // fprintf(gp, "set samples 10000\n");
        fprintf(gp, "plot '%s' using 5:2 with lines lc 'blue' title 'Drag', '%s' using 5:3 with lines lc 'red' title 'Lift'\n", filename_dat, filename_dat);
        fflush(gp); // Clean up data
    
        /*****************************************************************************/
    
        fprintf(gp, "set terminal pngcairo enhanced font 'Times New Roman,15' \n");
        fprintf(gp, "set output '%s'\n", filename_png_3);
        // fprintf(gp, "set multiplot\n");
        fprintf(gp, "set key left top\n");
        fprintf(gp, "set key font ',22'\n");
        fprintf(gp, "set term pngcairo size 1280, 960 font ',27'\n");
        // fprintf(gp, "set size ratio %.3f\n", size);
    
        fprintf(gp, "set lmargin screen 0.10\n");
        fprintf(gp, "set rmargin screen 0.90\n");
        fprintf(gp, "set tmargin screen 0.90\n");
        fprintf(gp, "set bmargin screen 0.15\n");
    
        fprintf(gp, "set xrange [%d:%d]\n", x_min, x_max);
        fprintf(gp, "set xlabel '%s'offset 0.0,0\n", xxlabel);
        fprintf(gp, "set yrange [%.3f:%.3f]\n", y_min, y_max_loadcell);
        fprintf(gp, "set ylabel '%s'offset 1.0,0.0\n", yylabel);
        fprintf(gp, "set title 'All sensors : %s [deg]'\n", label);
    
        // fprintf(gp, "set samples 10000\n");
        fprintf(gp, "plot '%s' using 5:4 with lines lc 'black' title 'loadcell', '%s' using 5:2 with lines lc 'blue' title 'Drag', '%s' using 5:3 with lines lc 'red' title 'Lift'\n", filename_dat, filename_dat, filename_dat);
        fflush(gp); // Clean up data
    
        fprintf(gp, "exit\n"); // Quit gnuplot
    
        pclose(gp);
    
        printf("01\t[%s]\tsuccess\n", angle);
    }
        
\end{lstlisting}

}

\end{document}