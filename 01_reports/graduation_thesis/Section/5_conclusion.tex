\section{結言}
本研究では,回流水槽を用いたタイヤモデルに加わる作用力測定実験について,
使用する作用力測定装置の揚抗力の同時測定における
性能評価および作用力へと換算する際の補正理論の構成とその有効性を示すことを目的とした.

作用力測定実験では,2ゲージ法を採用した2組のひずみセンサからそれぞれ
抗力方向と揚力方向の2方向の作用力の同時測定を行うこととなり,
そのときに得られるひずみセンサからの出力電圧は
回流水槽への実験装置の取付,ひずみセンサ貼付け時の不正確性といった人為的操作により
引き起こされる座標系の不一致によって大きな影響を及ぼされることがわかった.

そこで,作用力測定装置に複数の角度からロードセルによる作用力を与え,
ロードセルの出力電圧と作用力測定装置に取り付けられたひずみセンサの出力電圧の関係から
出力電圧勾配を定義し,作用力方向による出力電圧勾配の変化について評価を行うことを試みた.

作用力測定実化を行う際には,特に作用力測定装置のひずみセンサを基準とした座標系Aと
回流水槽の水流を基準とした水槽座標系に対する座標軸の回転角が大きく影響すると考えられる.
また,校正実験装置を用いて作用力測定装置の性能評価を行う際には,
校正装置を基準とした座標系Bと水槽座標系のオフセット距離が大きな影響を及ぼす事が考えられる.

ここで,上述の座標系の回転角とオフセット距離に注目し,それらの関係から補正理論を構成した.
また,テストデータへの適用による補正理論の評価を行った.

最後に,性能評価実験の結果に構成した補正理論を適用し,適用前と適用後について
理論値との差をRMS誤差によって評価を行った.

本研究を通して得られた知見は以下の通りである.

\begin{enumerate}[(a)]
    \item   水槽座標系と座標系Bのオフセット距離$\Delta x$,$\Delta y$が既知であるとき,実験結果から
            水槽座標系と座標系Aの回転角$\theta_x$,$\theta_y$,及び作用力測定装置に取り付けられた
            ひずみセンサの取付角$\phi_s$を推定することができることがわかった.\\
    \item   補正前の実験値と補正後の補正値について,理論値とのRMS誤差を評価すると,補正後のRMS誤差の値が抗力方向・揚力方向について小さいことがわかった.
            したがって,構成した補正理論は有効であることがわかった.\\
    \item   補正値の正味出力電圧による評価について,周期的な変動がみられた.
            これは,作用力測定装置のひずみセンサが2ゲージ法を用いて取り付けられていることによる影響と考えられ,
            その影響を考慮した補正理論の検討が必要であると考えられる.
\end{enumerate}

今後の展望として,正味出力電圧の変動の原因を突き止め,その影響を考慮した補正理論を追加しすることで,
実験結果からより精度の高い補正結果を得ることのできる理論の構築を目標として研究に取り組んでいきたい.
