\section{結言}
抗力方向と揚力宝庫の2方向の作用力の同時測定を目的とした
作用力測定実験および校正実験の際に得られるひずみセンサからの出力電圧は
回流水槽への実験装置の取付,ひずみセンサ貼付け時の不正確性といった人為的操により
引き起こされる座標系の不一致によって大きな影響を及ぼされることがわかった.
ここで,作用力測定装置の性能評価と回流水槽を用いた作用力測定実験の結果へと
活用できる理論の作成とその有効性を示すことを目的として研究を行った.
特に大きな要因になる座標系の回転とオフセット距離に注目し,その位置関係から補正理論を作成した.
作成した補正理論により,水槽座標系と座標系Bのオフセット距離$\Delta x$,$\Delta y$が既知であるとき,実験結果から
水槽座標系と座標系Aの回転角$\theta_x$,$\theta_y$を推定することができ,
その値を用いて水槽座標系における正味出力電圧勾配$v_x$,$v_y$を推定することができることがわかった.
しかしながら,Fig.の正味出力電圧勾配をみると,周期的な変動がみられる.
ここで,作用力測定装置は2ゲージ法を用いてひずみの計測を行っているため,
その影響により正味出力電圧勾配の振動が発生している可能性が考えられる.
今後の展望として,正味出力電圧の変動の原因を突き止め,その影響を考慮した補正理論を追加しすることで,
実験結果からより精度の高い補正結果を得ることのできる理論の構築を目標として研究に取り組んでいきたい.
