
\maketitle
\section*{概要}

本研究では,ひずみセンサを使用した揚抗力の同時測定を行う際に発生する現象の理解
とその補正方法の確立を目的とし,揚抗力測定装置を複数の角度から



\newpage
\setcounter{tocdepth}{2}
\tableofcontents
\newpage

\section*{記号表}
\begin{flalign*}
    (x, y) \quad &: \quad 水流に対する座標系,正規座標系 \\
    (x', y') \quad &: \quad 作用力測定装置の座標系,座標系[1]\\
    (x'', y'') \quad &: \quad 校正実験装置の座標系,座標系[2] \\
    V_{d} \quad &: \quad 作用力測定実験から得た抗力方向における出力電圧 \; \mathrm{[V]} & \\
    V_{l} \quad &: \quad 作用力測定実験から得た揚力方向における出力電圧 \; \mathrm{[V]} & \\
    F_{x} \quad &: \quad 正規座標系\;(x,y)\;について抗力方向における荷重 \; \mathrm{[N]} & \\
    F_{y} \quad &: \quad 正規座標系\;(x,y)\;について揚力方向における荷重 \; \mathrm{[N]} & \\
    \theta_x \quad &: \quad 正規座標系\; x軸と座標系[1]\; x'軸の角度 \mathrm{[deg]}\\
    \theta_y \quad &: \quad 正規座標系\; y軸と座標系[1]\; y'軸の角度 \mathrm{[deg]}\\
    \Delta x \quad &: \quad 正規座標系\; x軸と座標系[2]\; x''軸のy方向の距離 \mathrm{[mm]}\\
    \Delta y \quad &: \quad 正規座標系\; y軸と座標系[2]\; y''軸のx方向の距離 \mathrm{[mm]}\\
    v_{d} \quad &: \quad 基礎実験結果から得た抗力方向における出力電圧勾配 \; \mathrm{[V/V]} & \\
    v_{l} \quad &: \quad 基礎実験結果から得た揚力方向における出力電圧勾配 \; \mathrm{[V/V]} & \\
    v_{d'} \quad &: \quad 軸のオフセットについて補正を適用した抗力方向における出力電圧勾配 \; \mathrm{[V/V]} & \\
    v_{l'} \quad &: \quad 軸のオフセットについて補正を適用した揚力方向における出力電圧勾配 \; \mathrm{[V/V]} & \\
    v_{x} \quad &: \quad 正規座標系\;(x,y)\;について抗力方向における出力電圧勾配 \; \mathrm{[V/V]} & \\
    v_{y} \quad &: \quad 正規座標系\;(x,y)\;について揚力方向における出力電圧勾配 \; \mathrm{[V/V]} & \\
    \\
    3.x\; &\quad \textgt{抗力・揚力における出力電圧勾配の理論式}&\\
    v_{x\; Theory} \quad &: \quad 理論上の抗力方向における出力電圧勾配&\\
    v_{y\; Theory} \quad &: \quad 理論上の揚力方向における出力電圧勾配&\\
    \omega \quad &: \quad 角度\\
    \\
    3.2\; &\quad \textgt{座標系の回転における補正理論}&\\
    F_1 \quad &: \quad 作用力& \\
    \theta_{x1} \quad &: \quad 正規座標系\; x軸と座標系[1]\; x'軸の角度\\
    \theta_{y1} \quad &: \quad 正規座標系\; y軸と座標系[1]\; y'軸の角度\\
    F_{1x} \quad &: \quad x軸方向作用力& \\
    F_{1y} \quad &: \quad y軸方向作用力& \\
    F_{1x'} \quad &: \quad x'軸方向作用力& \\
    F_{1y'} \quad &: \quad y'軸方向作用力& \\
\end{flalign*}

\newpage
\section{序論}