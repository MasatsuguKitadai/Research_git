
\maketitle
\section*{概要}

本研究では,ひずみセンサを使用した揚抗力の同時測定を行う際に発生する現象の理解
とその補正方法の確立を目的とし,揚抗力測定装置に複数の角度から作用力を与えることによる
出力電圧の変化を測定するとともに,作用力測定装置の性能評価および補正理論の構成を行った.
実験結果から作用力測定装置の設置や,ひずみセンサの取付時の人為的操作によって
発生する誤差の影響を検討し,補正理論を構成・適用した結果,測定結果を理論値へと近づけることができた.
したがって,作用力測定装置を使用する上での人為的操作による誤差は
構成した補正理論を用いることで,それぞれの誤差を推定することができ,
実験結果を任意の座標系における出力結果へと変換することができる可能性を示すことができた.

\newpage
\setcounter{tocdepth}{3}
\tableofcontents
\newpage

% \section*{記号表}
% \begin{flalign*}
%     (x, y) \quad &: \quad 水流に対する座標系,正規座標系 \\
%     (x', y') \quad &: \quad 作用力測定装置の座標系,座標系[1]\\
%     (x'', y'') \quad &: \quad 校正実験装置の座標系,座標系[2] \\
%     V_{d} \quad &: \quad 作用力測定実験から得た抗力方向における出力電圧 \; \mathrm{[V]} & \\
%     V_{l} \quad &: \quad 作用力測定実験から得た揚力方向における出力電圧 \; \mathrm{[V]} & \\
%     F_{x} \quad &: \quad 正規座標系\;(x,y)\;について抗力方向における荷重 \; \mathrm{[N]} & \\
%     F_{y} \quad &: \quad 正規座標系\;(x,y)\;について揚力方向における荷重 \; \mathrm{[N]} & \\
%     \theta_x \quad &: \quad 正規座標系\; x軸と座標系[1]\; x'軸の角度 \mathrm{[deg]}\\
%     \theta_y \quad &: \quad 正規座標系\; y軸と座標系[1]\; y'軸の角度 \mathrm{[deg]}\\
%     \Delta x \quad &: \quad 正規座標系\; x軸と座標系[2]\; x''軸のy方向の距離 \mathrm{[mm]}\\
%     \Delta y \quad &: \quad 正規座標系\; y軸と座標系[2]\; y''軸のx方向の距離 \mathrm{[mm]}\\
%     v_{d} \quad &: \quad 基礎実験結果から得た抗力方向における出力電圧勾配 \; \mathrm{[V/V]} & \\
%     v_{l} \quad &: \quad 基礎実験結果から得た揚力方向における出力電圧勾配 \; \mathrm{[V/V]} & \\
%     v_{d'} \quad &: \quad 軸のオフセットについて補正を適用した抗力方向における出力電圧勾配 \; \mathrm{[V/V]} & \\
%     v_{l'} \quad &: \quad 軸のオフセットについて補正を適用した揚力方向における出力電圧勾配 \; \mathrm{[V/V]} & \\
%     v_{x} \quad &: \quad 正規座標系\;(x,y)\;について抗力方向における出力電圧勾配 \; \mathrm{[V/V]} & \\
%     v_{y} \quad &: \quad 正規座標系\;(x,y)\;について揚力方向における出力電圧勾配 \; \mathrm{[V/V]} & \\
%     \\
%     3.x\; &\quad \textgt{抗力・揚力における出力電圧勾配の理論式}&\\
%     v_{x\; Theory} \quad &: \quad 理論上の抗力方向における出力電圧勾配&\\
%     v_{y\; Theory} \quad &: \quad 理論上の揚力方向における出力電圧勾配&\\
%     \omega \quad &: \quad 角度\\
%     \\
%     3.2\; &\quad \textgt{座標系の回転における補正理論}&\\
%     F_1 \quad &: \quad 作用力& \\
%     \theta_{x1} \quad &: \quad 正規座標系\; x軸と座標系[1]\; x'軸の角度\\
%     \theta_{y1} \quad &: \quad 正規座標系\; y軸と座標系[1]\; y'軸の角度\\
%     F_{1x} \quad &: \quad x軸方向作用力& \\
%     F_{1y} \quad &: \quad y軸方向作用力& \\
%     F_{1x'} \quad &: \quad x'軸方向作用力& \\
%     F_{1y'} \quad &: \quad y'軸方向作用力& \\
% \end{flalign*}
% \newpage

\section{序論}

深刻な地球温暖化を発端にエネルギー問題が叫ばれる昨今では,
再生可能エネルギーの活用や脱炭素社会に向けた取り組み\cite{2021_roadmap}が行われている.
特に自動車への関心は非常に大きく,欧州では2030年代にはガソリン,ディーゼル車の新車販売の禁止を掲げている.
それに伴い,日本でも2035年以降には乗用車の新車販売の10割を電動自動車にすることを目標に掲げている.
ここで,例に挙げた自動車や航空機,船舶等の輸送機械の設計および開発を行う際に重要視されるのが
流体による作用力である.特に進行を妨げる方向にはたらく力である抗力の低減は,
輸送機械の燃費向上を考える上では非常に重要な要素の1つであるといえる.
また,その法線方向にはたらく揚力は,航空機の性能を定める最重要項目であるといっても過言ではない.
したがって,エネルギー問題に対応するために,今後の輸送機械の設計,開発の過程において,
機体周りの流場とともに機体に加わる揚抗力の理解が必要不可欠であるといえる.

ここで,流場の理解においては,
風洞の内の流場を線状の白煙によって可視するスモークワイヤ法\cite{1994_smokewire}や,
対象物の表面にタフトを貼付け,その挙動から流れの方向や剥離点を調べる表面タフト法\cite{1994_taft},
トレーサ粒子を流体中に混合させ,その粒子群の撮影画像を解析することで流場を可視化する
PIV(Particle Image Velocimetry)\cite{1999_PIV}等の手法が挙げられる.
また,自動車や航空機といった流体機械の設計,開発の現場においては,長らく上述の手段を用いて評価が行われてきた.\cite{1974_automobile}
現在では,計算流体力学(Computational Fluid Dynamics : CFD)を用いた
コンピュータによる流体解析手法が著しい発展を遂げている.
風洞等を用いた大規模な実験を行う必要がなく,設計ソフトウェアを使用して作成した3Dモデルのみで
流場の解析を擬似的に行うことができるといった利点がある.
そのことから設計,開発の期間の短縮やコスト削減を期待することができ,
流体機械の設計,開発\cite{2013_aircraft}\cite{1996_ship}においては必要不可欠なツールとなった.
また,流場における解析結果の整合性を示すため,簡易モデルを用いた実験結果との比較,評価を行う研究\cite{2007_CFD_comparison}も行われている.

一方で,上述の流場を理解するための手法では対象物に加わる揚抗力を測定することは
困難であり,CFDシミュレーションによる,マクロな流れの解析や揚抗力についての推定結果は
信頼性を検討しなければならない.\cite{2001_CFD_kobe}
そのため,実際に実験を行うことで得られる知見は非常に大きく,
特に,実際の流場により発生する作用力の測定方法について研究を行うことには大きな価値があるといえる.

本研究の背景として,田中ら\cite{2019_master}の行った回流水槽を用いたタイヤモデル(供試体)の作用力測定実験にならって,
供試体にはたらく作用力を実験的手法を用いて計測を行うことが目標である.
本研究で用いる作用力測定装置はひずみセンサを用いて揚抗力を曲げひずみによって変動する電圧として測定しており,
タイヤモデルとホイールハウスモデルの関係性を調べることから,供試体が大きく変位することは避けるべきである.
したがって,供試体を支える部材は剛性が高く,発生するひずみ量が小さくなる.
そこで,作用力測定装置には高感度の半導体ひずみセンサを使用し,2アクティブゲージ法\cite{2006_strainsensor}を用いているため,
2倍の出力電圧を得ることができるといった特徴を持つ.
ここで,2組のひずみセンサを直角に取り付けることで,揚抗力の測定を同時に行うこととなるが,
その手法や問題点について検討する必要がある.

以上のことから,本研究ではひずみセンサを用いた揚抗力の測定手法の開発やその問題点について検討し,
作用力測定装置の性能評価および実験結果に対する補正理論の構成を目的とした.

本論文中では,第2章で本実験で使用した作用力測定装置および新たに製作した校正実験装置の詳細について述べる.
第3章では,作用力実験装置と回流水槽および校正実験装置の位置関係から構成される補正理論について述べている.
第4章では,実施した作用力測定装置の性能評価実験の詳細とその結果を示す.また,その結果を用いた評価を行う.
最後に第5章で本研究の総括を述べる.
