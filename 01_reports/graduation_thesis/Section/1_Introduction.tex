
\maketitle
\section*{概要}

本研究では,ひずみセンサを使用した揚抗力の同時測定を行う際に発生する現象の理解
とその補正方法の確立を目的とし,揚抗力測定装置に複数の角度から作用力を与えることによる
出力電圧の変化を測定し,作用力測定装置の性能評価および補正理論の作成を行った.
実験結果から作用力測定装置の設置や,ひずみセンサの取付時の人為的操作によって
発生する誤差の影響を検討し,補正理論を作成・適用した結果,測定結果を理論値へと近づけることができた.
したがって,作用力測定装置を使用する上での人為的操作による誤差は
作成した補正理論を用いることで,それぞれの誤差を推定することができ,
実験結果を任意の座標系における出力結果へと変換することができる可能性を示すことができた

\newpage
\setcounter{tocdepth}{2}
\tableofcontents
\newpage

% \section*{記号表}
% \begin{flalign*}
%     (x, y) \quad &: \quad 水流に対する座標系,正規座標系 \\
%     (x', y') \quad &: \quad 作用力測定装置の座標系,座標系[1]\\
%     (x'', y'') \quad &: \quad 校正実験装置の座標系,座標系[2] \\
%     V_{d} \quad &: \quad 作用力測定実験から得た抗力方向における出力電圧 \; \mathrm{[V]} & \\
%     V_{l} \quad &: \quad 作用力測定実験から得た揚力方向における出力電圧 \; \mathrm{[V]} & \\
%     F_{x} \quad &: \quad 正規座標系\;(x,y)\;について抗力方向における荷重 \; \mathrm{[N]} & \\
%     F_{y} \quad &: \quad 正規座標系\;(x,y)\;について揚力方向における荷重 \; \mathrm{[N]} & \\
%     \theta_x \quad &: \quad 正規座標系\; x軸と座標系[1]\; x'軸の角度 \mathrm{[deg]}\\
%     \theta_y \quad &: \quad 正規座標系\; y軸と座標系[1]\; y'軸の角度 \mathrm{[deg]}\\
%     \Delta x \quad &: \quad 正規座標系\; x軸と座標系[2]\; x''軸のy方向の距離 \mathrm{[mm]}\\
%     \Delta y \quad &: \quad 正規座標系\; y軸と座標系[2]\; y''軸のx方向の距離 \mathrm{[mm]}\\
%     v_{d} \quad &: \quad 基礎実験結果から得た抗力方向における出力電圧勾配 \; \mathrm{[V/V]} & \\
%     v_{l} \quad &: \quad 基礎実験結果から得た揚力方向における出力電圧勾配 \; \mathrm{[V/V]} & \\
%     v_{d'} \quad &: \quad 軸のオフセットについて補正を適用した抗力方向における出力電圧勾配 \; \mathrm{[V/V]} & \\
%     v_{l'} \quad &: \quad 軸のオフセットについて補正を適用した揚力方向における出力電圧勾配 \; \mathrm{[V/V]} & \\
%     v_{x} \quad &: \quad 正規座標系\;(x,y)\;について抗力方向における出力電圧勾配 \; \mathrm{[V/V]} & \\
%     v_{y} \quad &: \quad 正規座標系\;(x,y)\;について揚力方向における出力電圧勾配 \; \mathrm{[V/V]} & \\
%     \\
%     3.x\; &\quad \textgt{抗力・揚力における出力電圧勾配の理論式}&\\
%     v_{x\; Theory} \quad &: \quad 理論上の抗力方向における出力電圧勾配&\\
%     v_{y\; Theory} \quad &: \quad 理論上の揚力方向における出力電圧勾配&\\
%     \omega \quad &: \quad 角度\\
%     \\
%     3.2\; &\quad \textgt{座標系の回転における補正理論}&\\
%     F_1 \quad &: \quad 作用力& \\
%     \theta_{x1} \quad &: \quad 正規座標系\; x軸と座標系[1]\; x'軸の角度\\
%     \theta_{y1} \quad &: \quad 正規座標系\; y軸と座標系[1]\; y'軸の角度\\
%     F_{1x} \quad &: \quad x軸方向作用力& \\
%     F_{1y} \quad &: \quad y軸方向作用力& \\
%     F_{1x'} \quad &: \quad x'軸方向作用力& \\
%     F_{1y'} \quad &: \quad y'軸方向作用力& \\
% \end{flalign*}
% \newpage

\section{序論}

深刻な地球温暖化を発端にエネルギー問題が叫ばれる昨今では,
再生可能エネルギーの活用や脱炭素社会に向けた取り組みが行われている.(地域脱炭素ロードマップ)
特に自動車への関心は非常に大きく,欧州では2030年代にはガソリン,ディーゼル車の新車販売の禁止を掲げている.
それに伴い,日本でも2035年以降には乗用車の新車販の10割を電動自動車にすることを目標に掲げている.
ここで,例に挙げた自動車や航空機,船舶等の輸送機械の設計および開発を行う際に重要視されるのが
流体による作用力である.特に進行を妨げる方向にはたらく力である抗力の低減は,
輸送機械の燃費向上を考える上では非常に重要なパラメータの1つであるといえる.
また,その法線方向にはたらく揚力は,航空機の性能を定める最重要項目であるといっても過言ではない.
したがって,エネルギー問題に対応するためには,今後の輸送機械の設計,開発の過程において,
機体の周りの流場とともに機体に加わる抗力および揚力といった作用力の理解が必要不可欠であるといえる.