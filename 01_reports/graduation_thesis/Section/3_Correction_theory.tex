\section{校正理論}
作用力測定装置から得た抗力方向および揚力方向における出力電圧$V_D$,$V_L$を
正規座標系の$x$軸方向および$y$軸方向の荷重$F_x$,$F_y$に換算する際に,
出力電圧$V_D$,$V_L$と$F_x$,$F_y$の関係性を明らかにするための校正実験を
行う必要がある.
校正実験によって得られた結果を用いて関係性を明らかにするための校正理論について述べる.

\subsection{作用力測定装置と校正実験装置の関係}
はじめに,作用力測定装置と校正実験装置の関係について説明する.

作用力測定装置と校正実験装置の設置位置によって校正実験結果は大きく変動するため,
その影響を考慮し,補正処理を行う必要がある.
このとき以下のような要因が,校正実験結果への影響を与えていると考えられる.

\begin{enumerate}[(1)]
    \item 作用力測定装置にひずみセンサが正確に取り付けることが難しい
    \item 作用力測定装置が回流水槽に正確に設置することが難しい
    \item 作用力測定装置と校正装置の回転軸を一致させることが難しい
\end{enumerate}

ここで,水流に対する座標系を正規座標系 $(x,y)$,
作用力測定装置の座標系を座標系(1) $(x',y')$,
校正装置の座標系を座標系(2) $(x,'' y'')$とする.

このとき,(1) 作用力測定装置にひずみセンサを正確に取り付けることが難しいこと,
(2) 作用力測定装置が回流水槽に正確に設置することが難しいことから,
座標系[1]は正規座標系に対して
$x'$軸は$x$軸から$\theta_x1$,$y'$軸は$y$軸から$\theta_y1$だけ回転している.
また,座標系[2]は正規座標系に対して
$x''$軸は$x$軸から$y$方向に$\Delta x$,
$y''$軸は$y$軸から$x$方向に$\Delta y$だけオフセットを持つ状態となる.

\newpage

\subsection{座標系の回転における補正理論}

正規座標系と座標系[1]の回転における補正理論を説明する.
ここでは,座標系のオフセットはない($\Delta x = 0$,$\Delta y = 0$)として考える.
上述の通り正規座標系と座標系[1]について,以下のFig.のように回転角$\theta_{x1}$,$\theta_{y1}$を持つ.
ここで,作用力$F_1$を与えるとそれぞれの方向に作用力$F_{1x}$,$F_{1y}$,$F_{1x'}$,$F_{1y'}$が加わる.
このとき,作用力測定装置から得られる電圧$V_d$,$V_l$は作用力$F_{1x'}$,$F_{1y'}$に起因するものであるため,
$F_{1x}$と$F_{1x'}$および$F_{1y}$と$F_{1y'}$の関係性を明らかにしなければならない.

\subsection{回転角$\theta_{x1}$,$\theta_{y1}$の算出}
はじめに,回転角$\theta_{x1}$,$\theta_{x2}$を算出する.
理論式における$v_{x\;Theory}$及び$v_{y\;Theory}$は正弦波とその位相差で表すことができる.
したがって,$F_{1x}$,$F_{1y}$,$F_{1x'}$,$F_{1y'}$も同様に正弦波

\newpage

\subsection{座標系のオフセットにおける補正理論}

正規座標系と座標系[2]のオフセットの補正理論を説明する.


\subsection{複合状態における補正理論}

\subsection{推定理論}