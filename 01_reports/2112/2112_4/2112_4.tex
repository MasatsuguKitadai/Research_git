\documentclass[twocolumn,a4j]{jsarticle}
\setlength{\topmargin}{-20.4cm}
\setlength{\oddsidemargin}{-10.4mm}
\setlength{\evensidemargin}{-10.4mm}
\setlength{\textwidth}{18cm}
\setlength{\textheight}{26cm}

\usepackage[top=15truemm,bottom=25truemm,left=15truemm,right=15truemm]{geometry}
\usepackage[latin1]{inputenc}
\usepackage{amsmath}
\usepackage{amsfonts}
\usepackage{amssymb}
\usepackage[dvipdfmx]{graphicx}
\usepackage[dvipdfmx]{color}
\usepackage{listings}
\usepackage{listings,jvlisting}
\usepackage{geometry}
\usepackage{framed}
\usepackage{color}
\usepackage[dvipdfmx]{hyperref}
\usepackage{ascmac}
\usepackage{enumerate}
\usepackage{tabularx}
\usepackage{cancel}
\usepackage{scalefnt}

\renewcommand{\figurename}{Fig.}
\renewcommand{\tablename}{Table }

\lstset{
basicstyle={\ttfamily},
identifierstyle={\small},
commentstyle={\smallitshape},
keywordstyle={\small\bfseries},
ndkeywordstyle={\small},
stringstyle={\small\ttfamily},
frame={tb},
breaklines=true,
columns=[l]{fullflexible},
xrightmargin=0zw,
xleftmargin=3zw,
numberstyle={\scriptsize},
stepnumber=1,
numbersep=1zw,
lineskip=-0.5ex
}

\makeatletter
\def\@maketitle
{
\begin{center}
{\LARGE \@title \par}
\end{center}
\begin{flushright}
{\large 報告書 NO.08 - 4\quad\@date\quad\@author}
\end{flushright}
\par\vskip 1.5em
}
\makeatother

\setcounter{tocdepth}{3}

\author{来代 勝胤}
\title{令和3年度 12月 第4週 報告書}
\date{2021/12/23}

\begin{document}
\columnseprule=0.1mm

\maketitle
\section*{報告内容}
\begin{enumerate}[1.]
    \item 進捗状況
    \item 再実験の実施
    \item 理論値の算出
    \item FFTの適用
    \item 位相角の算出
\end{enumerate}

\section{進捗状況}
今週は,測定したデータ処理を行うプログラムの作成を行った。
また、FFTを適用し位相角を求めるため、再度実験を行い
そのデータと理論値との差異を算出した。

\section{再実験の実施}
FFTを用いるためにデータ数を2の乗数に合わせなければならないことから、
前回行った実験と同様の手法で再実験を行うこととした。
今回は、22.5度刻みの計16方向から、ひずみセンサおよびロードセルの出力電圧の測定を行った。
実験結果を以下のTable 1および、Fig.1、Fig.2に示す。\\

\begin{table}[htbp]
    \begin{center}
        \caption{Result summary}
        \begin{tabular}{|p{20mm}|p{20mm}|p{20mm}|p{20mm}|}
            \hline
            \multicolumn{1}{|c|}{\textgt{Angle [deg]}} & \multicolumn{1}{|c|}{\textgt{Drag [V/V]}} & \multicolumn{1}{|c|}{\textgt{Lift [V/V]}} & \multicolumn{1}{|c|}{\textgt{Net [V/V]}} \\ \hline
            \multicolumn{1}{|c|}{0.0}                  & \multicolumn{1}{|r|}{-0.629891}           & \multicolumn{1}{|r|}{\textgt{0.096225}}   & \multicolumn{1}{|r|}{\textgt{0.637198}}  \\ \hline
            \multicolumn{1}{|c|}{22.5}                 & \multicolumn{1}{|r|}{-0.631390}           & \multicolumn{1}{|r|}{\textgt{-0.135281}}  & \multicolumn{1}{|r|}{\textgt{0.645720}}  \\ \hline
            \multicolumn{1}{|c|}{45.0}                 & \multicolumn{1}{|r|}{-0.505268}           & \multicolumn{1}{|r|}{\textgt{-0.400433}}  & \multicolumn{1}{|r|}{\textgt{0.644703}}  \\ \hline
            \multicolumn{1}{|c|}{67.5}                 & \multicolumn{1}{|r|}{-0.305154}           & \multicolumn{1}{|r|}{\textgt{-0.564455}}  & \multicolumn{1}{|r|}{\textgt{0.641660}}  \\ \hline
            \multicolumn{1}{|c|}{90.0}                 & \multicolumn{1}{|r|}{-0.065062}           & \multicolumn{1}{|r|}{\textgt{-0.626712}}  & \multicolumn{1}{|r|}{\textgt{0.630080}}  \\ \hline
            \multicolumn{1}{|c|}{112.5}                & \multicolumn{1}{|r|}{0.200668}            & \multicolumn{1}{|r|}{\textgt{-0.613089}}  & \multicolumn{1}{|r|}{\textgt{0.645094}}  \\ \hline
            \multicolumn{1}{|c|}{135.0}                & \multicolumn{1}{|r|}{0.368146}            & \multicolumn{1}{|r|}{\textgt{-0.531723}}  & \multicolumn{1}{|r|}{\textgt{0.646731}}  \\ \hline
            \multicolumn{1}{|c|}{157.5}                & \multicolumn{1}{|r|}{0.575943}            & \multicolumn{1}{|r|}{\textgt{-0.321693}}  & \multicolumn{1}{|r|}{\textgt{0.659694}}  \\ \hline
            \multicolumn{1}{|c|}{180.0}                & \multicolumn{1}{|r|}{0.632274}            & \multicolumn{1}{|r|}{\textgt{-0.079331}}  & \multicolumn{1}{|r|}{\textgt{0.637231}}  \\ \hline
            \multicolumn{1}{|c|}{202.5}                & \multicolumn{1}{|r|}{0.613635}            & \multicolumn{1}{|r|}{\textgt{0.171695}}   & \multicolumn{1}{|r|}{\textgt{0.637203}}  \\ \hline
            \multicolumn{1}{|c|}{225.0}                & \multicolumn{1}{|r|}{0.539625}            & \multicolumn{1}{|r|}{\textgt{0.364962}}   & \multicolumn{1}{|r|}{\textgt{0.651454}}  \\ \hline
            \multicolumn{1}{|c|}{247.5}                & \multicolumn{1}{|r|}{0.319874}            & \multicolumn{1}{|r|}{\textgt{0.550715}}   & \multicolumn{1}{|r|}{\textgt{0.636873}}  \\ \hline
            \multicolumn{1}{|c|}{270.0}                & \multicolumn{1}{|r|}{0.053645}            & \multicolumn{1}{|r|}{\textgt{0.634929}}   & \multicolumn{1}{|r|}{\textgt{0.637191}}  \\ \hline
            \multicolumn{1}{|c|}{292.5}                & \multicolumn{1}{|r|}{-0.179625}           & \multicolumn{1}{|r|}{\textgt{0.619397}}   & \multicolumn{1}{|r|}{\textgt{0.644917}}  \\ \hline
            \multicolumn{1}{|c|}{315.0}                & \multicolumn{1}{|r|}{-0.407471}           & \multicolumn{1}{|r|}{\textgt{0.503615}}   & \multicolumn{1}{|r|}{\textgt{0.647812}}  \\ \hline
            \multicolumn{1}{|c|}{337.5}                & \multicolumn{1}{|r|}{-0.575843}           & \multicolumn{1}{|r|}{\textgt{0.304549}}   & \multicolumn{1}{|r|}{\textgt{0.651418}}  \\ \hline
            \multicolumn{1}{|c|}{Average}              & \multicolumn{1}{|r|}{0.000257}            & \multicolumn{1}{|r|}{\textgt{-0.001664}}  & \multicolumn{1}{|r|}{\textgt{0.643436}}  \\ \hline
        \end{tabular}
    \end{center}
\end{table}

\newpage

\begin{figure}[htbp]
    \footnotesize
    \begin{center}
        \includegraphics[width=88mm]{../images_2/05/05_summary-wave.png}
        \caption{Summary of gradient value}
        \includegraphics[width=88mm]{../images_2/05/05_summary-outputvoltage.png}
        \caption{Summary of net voltage value}
    \end{center}
\end{figure}

実験結果から算出した正味出力電圧の分散及び標準偏差を以下のTable 2 に示す.

\begin{table}[htbp]
    \begin{center}
        \caption{}
        \begin{tabular}{|p{20mm}|p{20mm}|}
            \hline
            \multicolumn{1}{|c|}{\textgt{分散}}     & \multicolumn{1}{|c|}{0.000051} \\ \hline
            \multicolumn{1}{|c|}{\textgt{標準偏差}} & \multicolumn{1}{|c|}{0.007133} \\ \hline
        \end{tabular}
    \end{center}
\end{table}

ここで,Table.1の正味出力電圧と標準偏差を比較すると,

\newpage

\section{理論値の算出}
電圧の測定実験において、抗力および揚力の出力電圧は正弦波の位相がそれぞれ$-\pi/2$、$\pi$だけ
進んだ波形になると考えられる。

\begin{eqnarray*}
    \mathrm{Drag Voltage} &=& A \sin\left(\omega t - \frac{\pi}{2}\right)\\
    \mathrm{Lift Voltage} &=& A \sin\left(\omega t + \pi\right)\\
\end{eqnarray*}

ここで、Table 1の正味の出力電圧の平均値を振幅とし、
抗力および揚力についてそれぞれ位相が進んだ正弦波を作成した。
その算出結果を以下のFig.3に示す。

\begin{figure}[htbp]
    \footnotesize
    \begin{center}
        \includegraphics[width=88mm]{../images_2/20/20_adjust-value.png}
        \caption{value}
    \end{center}
\end{figure}

\newpage

\section{FFTの適用}


\begin{figure}[htbp]
    \footnotesize
    \begin{center}
        \includegraphics[width=80mm]{../images_2/07/07-1.png}
        \caption{FFT Result [Drag]}
        \includegraphics[width=80mm]{../images_2/07/07-2.png}
        \caption{FFT Result [Lift]}
    \end{center}
\end{figure}

\section{位相角の算出}

\end{document}