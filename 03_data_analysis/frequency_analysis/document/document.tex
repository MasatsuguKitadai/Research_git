\documentclass[a4paper]{jsarticle}
\setlength{\topmargin}{-20.4cm}
\setlength{\oddsidemargin}{-10.4mm}
\setlength{\evensidemargin}{-10.4mm}
\setlength{\textwidth}{18cm}
\setlength{\textheight}{26cm}

\usepackage[top=15truemm,bottom=25truemm,left=20truemm,right=20truemm]{geometry}
\usepackage[latin1]{inputenc}
\usepackage{amsmath}
\usepackage{amsfonts}
\usepackage{amssymb}
\usepackage[dvipdfmx]{graphicx}
\usepackage[dvipdfmx]{color}
\usepackage{listings}
\usepackage{listings,jvlisting}
\usepackage{geometry}
\usepackage{framed}
\usepackage{color}
\usepackage[dvipdfmx]{hyperref}
\usepackage{ascmac}
\usepackage{enumerate}
\usepackage{tabularx}
\usepackage{cancel}
\usepackage{scalefnt}

\renewcommand{\figurename}{fig.}
\renewcommand{\tablename}{table }
\newcommand{\redunderline}[1]{\textcolor{BrickRed}{\underline{\textcolor{black}{#1}}}} 

\hypersetup{
	colorlinks=false, % リンクに色をつけない設定
	bookmarks=true, % 以下ブックマークに関する設定
	bookmarksnumbered=true,
	pdfborder={0 0 0},
	bookmarkstype=toc
}

\lstset{
basicstyle={\ttfamily},
identifierstyle={\small},
commentstyle={\smallitshape},
keywordstyle={\small\bfseries},
ndkeywordstyle={\small},
stringstyle={\small\ttfamily},
frame={tb},
breaklines=true,
columns=[l]{fullflexible},
xrightmargin=0zw,
xleftmargin=3zw,
numberstyle={\scriptsize},
stepnumber=1,
numbersep=1zw,
lineskip=-0.5ex
}



\author{}
\title{C言語}
\date{}

\begin{document}
\maketitle
\section{用語}
\subsubsection*{バッファ}
バッファとは、データが 1 つの場所から別の場所に転送する際に、データを一時的に保持するメモリストレージ領域のことをいいます。\\
プログラムがデータをバッファに書き込もうとした際に、データの量がメモリバッファのストレージ容量を超えてしまうと、隣接するメモリ位置を上書きしてしまいます。\\
この結果発生する事象をバッファオーバーフロー(バッファオーバーラン)と呼びます。\\
\section{ファイルの読み書き}

\begin{itembox}[l]{ファイルへの読み書き}
    \begin{enumerate}[(1)]
        \item 1文字の読み込み\\
              int fgetc(FILE *fp);
        \item 1文字の書き出し\\
              int fputc(int c, FILE *fp);
        \item 文字列の書き出し\\
        \item int fprintf(FILE *steram, char *format, ...);
    \end{enumerate}
\end{itembox}
\end{document}