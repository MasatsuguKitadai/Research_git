\documentclass[a4paper]{jsarticle}
\setlength{\topmargin}{-20.4cm}
\setlength{\oddsidemargin}{-10.4mm}
\setlength{\evensidemargin}{-10.4mm}
\setlength{\textwidth}{18cm}
\setlength{\textheight}{26cm}

\usepackage[top=15truemm,bottom=25truemm,left=20truemm,right=20truemm]{geometry}
\usepackage[latin1]{inputenc}
\usepackage{amsmath}
\usepackage{amsfonts}
\usepackage{amssymb}
\usepackage[dvipdfmx]{graphicx}
\usepackage{listings}
\usepackage{listings,jvlisting}
\usepackage{geometry}
\usepackage{framed}
\usepackage{color}
\usepackage[dvipdfmx]{hyperref}

\hypersetup{
	colorlinks=false, % リンクに色をつけない設定
	bookmarks=true, % 以下ブックマークに関する設定
	bookmarksnumbered=true,
	pdfborder={0 0 0},
	bookmarkstype=toc
}

\lstset{
basicstyle={\ttfamily},
identifierstyle={\small},
commentstyle={\smallitshape},
keywordstyle={\small\bfseries},
ndkeywordstyle={\small},
stringstyle={\small\ttfamily},
frame={tb},
breaklines=true,
columns=[l]{fullflexible},
xrightmargin=0zw,
xleftmargin=3zw,
numberstyle={\scriptsize},
stepnumber=1,
numbersep=1zw,
lineskip=-0.5ex
}

\author{}
\title{プログラミング環境の構築}
\date{}

\begin{document}
\maketitle

想定しているのは以下の環境です。
\begin{itemize}
    \item windows10
    \item vscode
    \item wsl
\end{itemize}

\section{\large Tex のセットアップ}
\subsection{Tex live のインストール}
\textcolor{blue}{\href{http://www.tug.org/texlive/acquire-netinstall.html}{ここ}} から、Tex Liveのセットアップファイルをダウンロードする。\\
「 install-tl-windows.exe 」をクリックするとファイルのダウンロードが始まる。\\
ダウンロードしたファイルを起動し、手順に沿って進める。(基本、Enter連続でOK)\\
※ インストールは長時間かかります。(3時間程度)
インストールが完了したら、PCを再起動する。
\subsection{環境変数の設定}
環境変数とは、
\subsection{VScodeの設定}
PCの再起動後、VScodeを起動する。
Extentions [Ctrl + Shift + x]から、LaTex workshop をインストールする。
その後、settings json を開く。
このmanualフォルダ内の「json\_forTex.txt」の内容をコピー&ペーストして保存する。
左下の歯車マークをクリックし、「Command Palette」 をクリックする。\\
出てきた検索欄に、「LaTex Workshop: Build with recipe」と入力し、「latexmk」を選択する。
terminal [Ctrl + Shift + @]から新規ターミナルを開く。\\
ターミナルで、「mktexler」と入力し「Enter」を押す。\\
その後、アプリを再起動する。\\
\subsection{Texの動作確認}
フォルダ内の「test.tex」を任意のフォルダにコピーし、VScodから「test.tex」を開き、
[Ctrl + Alt + b]でファイルのビルド(pdfファイルの生成のこと)を行う。\\
フォルダ内に、「(ファイル名).aux」、「(ファイル名).log」、「(ファイル名).pdf」が生成されれば、無事に環境が構築されている。
[Ctrl + Alt + v]でpdfファイルを表示することができる。
\end{document}