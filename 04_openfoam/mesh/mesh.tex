\documentclass[a4paper]{jsarticle}
\setlength{\topmargin}{-20.4cm}
\setlength{\oddsidemargin}{-10.4mm}
\setlength{\evensidemargin}{-10.4mm}
\setlength{\textwidth}{18cm}
\setlength{\textheight}{26cm}

\usepackage[top=15truemm,bottom=25truemm,left=20truemm,right=20truemm]{geometry}
\usepackage[latin1]{inputenc}
\usepackage{amsmath}
\usepackage{amsfonts}
\usepackage{amssymb}
\usepackage[dvipdfmx]{graphicx}
\usepackage{listings}
\usepackage{listings,jvlisting}
\usepackage{geometry}
\usepackage{framed}

\lstset{
basicstyle={\ttfamily},
identifierstyle={\small},
commentstyle={\smallitshape},
keywordstyle={\small\bfseries},
ndkeywordstyle={\small},
stringstyle={\small\ttfamily},
frame={tb},
breaklines=true,
columns=[l]{fullflexible},
xrightmargin=0zw,
xleftmargin=3zw,
numberstyle={\scriptsize},
stepnumber=1,
numbersep=1zw,
lineskip=-0.5ex
}

\makeatletter
\def\@maketitle
{
\begin{center}
{\LARGE \@title \par}
\end{center}
\begin{flushright}
{\large OpenFOAM NO.01\quad\@date\quad\@author}
\end{flushright}
\par\vskip 1.5em
}
\makeatother

\author{}
\title{メッシュの作成手順}
\date{}

\begin{document}
\maketitle
\section{\large メッシュ作成の流れ}
\begin{itemize}
    \item blockMesh
    \item surfaceFeatures
    \item decomposePar
    \item snappyHexMesh
\end{itemize}
\section{\large blockMesh → 計算領域の設定を行う}
該当するファイルは, "system/blockMeshDict"
\subsection{単位の設定}
単位は"m"で,デフォルトでは"1"に設定されている.(仮に"mm"に設定したい場合は,0.001に書き換える.)任意の単位に変更できるが,
他の計算の設定と混同しないように統一しておくこと(基本的には変更しない)が望ましい.
\begin{framed}
    \begin{center}
        convertToMeters 1;
    \end{center}
\end{framed}
\subsection{領域の設定}
"vertice"に設定する解析範囲にしたがって座標(8点)を記入する.
原点は,3Dモデルと対応する.
\begin{framed}
    \begin{center}
        {
            vertices\\
            (\\
            \qquad (-0.250 -0.750 -0.025)\\
            \qquad (0.250 -0.750 -0.025)\\
            \qquad (0.250 0.250 -0.025)\\
            \qquad (-0.250 0.250 -0.025)\\
            \qquad (-0.250 -0.750 0.475)\\
            \qquad (0.250 -0.750 0.475)\\
            \qquad (0.250 0.250 0.475)\\
            \qquad (-0.250 0.250 0.475)\\
            );
        }
    \end{center}
\end{framed}

また,その分割数を指定する.aa
\end{document}