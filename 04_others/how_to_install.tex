\documentclass[a4paper]{jsarticle}
\setlength{\topmargin}{-20.4cm}
\setlength{\oddsidemargin}{-10.4mm}
\setlength{\evensidemargin}{-10.4mm}
\setlength{\textwidth}{18cm}
\setlength{\textheight}{26cm}

\usepackage[top=15truemm,bottom=25truemm,left=20truemm,right=20truemm]{geometry}
\usepackage[latin1]{inputenc}
\usepackage{amsmath}
\usepackage{amsfonts}
\usepackage{amssymb}
\usepackage[dvipdfmx]{graphicx}
\usepackage{listings}
\usepackage{listings,jvlisting}
\usepackage{geometry}
\usepackage{framed}
\usepackage{color}
\usepackage[dvipdfmx]{hyperref}

\hypersetup{
	colorlinks=false, % リンクに色をつけない設定
	bookmarks=true, % 以下ブックマークに関する設定
	bookmarksnumbered=true,
	pdfborder={0 0 0},
	bookmarkstype=toc
}

\lstset{
basicstyle={\ttfamily},
identifierstyle={\small},
commentstyle={\smallitshape},
keywordstyle={\small\bfseries},
ndkeywordstyle={\small},
stringstyle={\small\ttfamily},
frame={tb},
breaklines=true,
columns=[l]{fullflexible},
xrightmargin=0zw,
xleftmargin=3zw,
numberstyle={\scriptsize},
stepnumber=1,
numbersep=1zw,
lineskip=-0.5ex
}

\author{}
\title{プログラミング環境の構築}
\date{}

\begin{document}
\maketitle

想定しているのは以下の環境です。
\begin{itemize}
    \item windows10
    \item vscode
    \item wsl
\end{itemize}

\section{\large Tex のセットアップ}
\subsection{Tex live のインストール}
\textcolor{blue}{\href{http://www.tug.org/texlive/acquire-netinstall.html}{ここ}} から、Tex Liveのセットアップファイルをダウンロードする。\\
「 install-tl-windows.exe 」をクリックするとファイルのダウンロードが始まる。\\
ダウンロードしたファイルを起動し、手順に沿って進める。(基本、Enter連続でOK)\\
※ インストールは長時間かかります。(3時間程度)
\subsection{VScode での設定}
拡張機能から、LaTex workshop をインストールする。\\
settings json を開く。(usr、workspace のどちらでも可)\\
このmanualフォルダにある json\_forTex.txt の内容をコピー&ペーストして保存する。
左下の歯車マークをクリックし、Command Palette をクリックする。\\
出てきた検索欄に、「LaTex Workshop: Build with recipe」と入力し、「latexmk」を選択する。\\
その後、アプリをリフレッシュする。
\section{\large blockMesh → 計算領域の設定を行う}
該当するファイルは, "system/blockMeshDict"
\subsection{単位の設定}
単位は"m"で,デフォルトでは"1"に設定されている.(仮に"mm"に設定したい場合は,0.001に書き換える.)任意の単位に変更できるが,
他の計算の設定と混同しないように統一しておくこと(基本的には変更しない)が望ましい.
\begin{framed}
    \begin{center}
        convertToMeters 1;
    \end{center}
\end{framed}
\subsection{領域の設定}
"vertice"に設定する解析範囲にしたがって座標(8点)を記入する.
原点は,3Dモデルと対応する.
\begin{framed}
    \begin{center}
        {
            vertices\\
            (\\
            \qquad (-0.250 -0.750 -0.025)\\
            \qquad (0.250 -0.750 -0.025)\\
            \qquad (0.250 0.250 -0.025)\\
            \qquad (-0.250 0.250 -0.025)\\
            \qquad (-0.250 -0.750 0.475)\\
            \qquad (0.250 -0.750 0.475)\\
            \qquad (0.250 0.250 0.475)\\
            \qquad (-0.250 0.250 0.475)\\
            );
        }
    \end{center}
\end{framed}

また,その分割数を指定する.aa
\end{document}